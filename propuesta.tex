\documentclass[10pt]{article}
\usepackage[utf8]{inputenc}
\usepackage[spanish]{babel}
\usepackage{hyperref}
\usepackage{geometry}
\geometry{a4paper, margin=1in}

\title{Predicción de Precios de Acciones mediante Redes Neuronales}
\author{Alumno: José Antonio Barrientos Sánchez}
\date{\today}

\begin{document}

\maketitle

El presente proyecto se centra en la predicción de precios en el mercado de acciones utilizando técnicas de redes neuronales. Para ello, se utilizará el conjunto de datos titulado \emph{Stock Market Data}, obtenido de Kaggle\footnote{\url{https://www.kaggle.com/datasets/paultimothymooney/stock-market-data}}. Este dataset contiene información histórica relevante del mercado, lo que permite analizar patrones y tendencias en el comportamiento de los precios.

El problema a resolver consiste en desarrollar un modelo capaz de predecir futuros valores de las acciones basándose en datos pasados. 

Este problema es muy interesante pues los mercados financieros son complicados, y las redes neuronales aprenden automáticamente las conexiones difíciles entre las variables, así que podemos apoyarnos de las RN con modelos como LSTM o GRU que pueden recordar información importante a lo largo del tiempo, lo que ayuda a prever tendencias futuras basadas en datos pasados.

\subsection*{Variables para el Modelo Predictivo de Valores de Acciones}

\begin{itemize}
    \item \textbf{Variables de entrada (features):}
    \begin{itemize}
        \item \textit{Open:} Precio de apertura de la acción.
        \item \textit{High:} Precio máximo alcanzado durante el período.
        \item \textit{Low:} Precio mínimo alcanzado durante el período.
        \item \textit{Volume:} Volumen de transacciones realizadas.
    \end{itemize}
    \item \textbf{Variable de salida (target):}
    \begin{itemize}
        \item \textit{Close:} Precio de cierre de la acción, o una versión desplazada de este valor para predecir el precio del siguiente período.
    \end{itemize}
\end{itemize}


\subsection*{Bibliografía Consultada}
\begin{itemize}
    \item Kaggle. \textit{Stock Market Data}. Recuperado de \url{https://www.kaggle.com/datasets/paultimothymooney/stock-market-data}
\end{itemize}

\end{document}
