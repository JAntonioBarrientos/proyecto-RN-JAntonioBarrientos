\documentclass[12pt]{article}
\usepackage[utf8]{inputenc}
\usepackage[spanish]{babel}
\usepackage{hyperref}
\usepackage{geometry}
\geometry{a4paper, margin=1in}

\title{Predicción de Precios de Acciones mediante Redes Neuronales}
\author{José Antonio Barrientos Sánchez}
\date{\today}

\begin{document}

\maketitle

\section*{Introducción}
El presente proyecto se centra en la predicción de precios en el mercado de acciones utilizando técnicas de redes neuronales. Para ello, se utilizará el conjunto de datos titulado \emph{Stock Market Data}, obtenido de Kaggle\footnote{\url{https://www.kaggle.com/datasets/paultimothymooney/stock-market-data}}. Este dataset contiene información histórica relevante del mercado, lo que permite analizar patrones y tendencias en el comportamiento de los precios.

El problema a resolver consiste en desarrollar un modelo capaz de predecir futuros valores de las acciones basándose en datos pasados. Este problema es muy interesante para este problema por varias razones:

\begin{itemize}
    \item Los mercados financieros son complicados, y las redes neuronales aprenden automáticamente las conexiones difíciles entre las variables.
    \item  Modelos como LSTM o GRU pueden recordar información importante a lo largo del tiempo, lo que ayuda a prever tendencias futuras basadas en datos pasados.
\end{itemize}



\section*{Bibliografía Consultada}
\begin{itemize}
    \item Kaggle. \textit{Stock Market Data}. Recuperado de \url{https://www.kaggle.com/datasets/paultimothymooney/stock-market-data}
\end{itemize}

\end{document}
